\chapter{Abstract}

Due to the ever-increasing popularity of websites and mobile applications, demands on Web servers continue to grow. Well-developed software can help to limit hardware cost while boosting performance. Event- and actor-driven development paradigms aim to depart from traditional modalities in order to better suit modern Web server applications. Various technologies already implement different approaches to these patterns. This thesis aims to elaborate upon how, how much and under which circumstances these technologies can increase Web server performance.

%Over the past years, websites have advanced from merely displaying content to representing interfaces for dynamic server-side applications of various scales; other environments like mobile platforms tend to use the same HTTP interfaces as well. To limit the cost of server hardware, various software-based approaches aim to maximise the number of simultaneous operations by shifting from the classic per-request threading model to more sophisticated concurrency patterns. This thesis presents and compares a number of different approaches to server-side concurrency implementations from the view of a developer. Typical use-cases for server-side information flow are contrived and evaluated regarding asynchronous processing. Patterns are then reviewed based on their performance in scalable high-throughput networking applications by the example of live applications as well as experimental settings.